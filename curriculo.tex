\documentclass[11pt, a4paper]{article}
\usepackage[utf8]{inputenc}
\usepackage[T1]{fontenc}
\usepackage[default]{opensans}
\usepackage[margin=0.5in]{geometry}
\usepackage{hyperref}
\usepackage{enumitem}
\usepackage{parskip}

\hypersetup{
    colorlinks=true,
    urlcolor=blue,
}

\setlist[itemize]{noitemsep, topsep=0pt, leftmargin=*}

\begin{document}

% --- Header ---
\begin{center}
    {\Huge \textbf{ALEX SOARES PRESTES}} \\[0.2cm]
    {\large Engenheiro de Machine Learning | Computação Científica | HPC} \\[0.2cm]
    {\small Bauru, SP | \href{mailto:alex.prestes@outlook.com.br}{alex.prestes@outlook.com.br} | (14) 99728-8656} \\
    {\small \href{https://linkedin.com/in/alexsprestes}{linkedin.com/in/alexsprestes} | \href{https://github.com/AlexPrestes}{github.com/AlexPrestes}}
\end{center}

\vspace{-5pt}

% --- Resumo ---
\section*{Resumo Profissional}

\vspace{-5pt}

Engenheiro de Machine Learning com trajetória técnica construída a partir de formação científica e atuação contínua em projetos acadêmicos e aplicados. Especialista em engenharia de pipelines de dados, otimização de performance em Python e computação científica, com foco em problemas de alta complexidade e escala. Experiência prática em construção de pipelines de Machine Learning, validação estatística, modelagem em grafos e integração com sistemas corporativos (SAP Business One). Bacharel em Física (USP), com base sólida em álgebra linear, processamento eficiente e métodos numéricos. Atualmente cursando MBA em Data Science \& Analytics (USP/ESALQ), atuo no desenvolvimento de soluções ponta a ponta — da concepção metodológica à implementação, validação e documentação reprodutível.

\vspace{-5pt}


% --- Experiência ---
\section*{Experiência Profissional}

\vspace{-5pt}


\subsection*{Engenheiro de Machine Learning — Consultoria Técnica (Nov/2025)}
\textit{Remoto, Brasil}
\begin{itemize}
    \item Atuação na otimização de pipeline do mestrado de \textbf{link prediction em grafos biológicos (PPI)}.
    \item Modelagem do pipeline (ingestão, amostragem, treino e validação), obtendo ganho de \textbf{163x em performance} e \textbf{329x em escalabilidade}.
    \item Otimização vetorial e uso de \textbf{álgebra linear esparsa} para cálculo de features estruturais em \textbf{34 milhões de pares de nós}, com tempo total de execução de \textbf{4,4 segundos}, respeitando restrições de hardware (notebook i5, 10GB RAM).
    \item Desenvolvimento de \textbf{K-Fold customizado para grafos}, garantindo ausência de data leakage e avaliação metodologicamente correta.
    \item Implementação de validação estatística (teste t pareado), documentação integral do pipeline e garantia de \textbf{reprodutibilidade e auditabilidade científica}.
\end{itemize}

\vspace{-5pt}


\subsection*{Analista de Dados – Gemmini (Dez/2024 – Fev/2025)}
\begin{itemize}
    \item Atuação no suporte técnico à transição para o \textbf{SAP Business One}, garantindo consistência e integridade dos dados logísticos.
    \item Desenvolvimento de automações em \textbf{Python e Excel} para relatórios operacionais e indicadores de estoque.
    \item Suporte operacional à geração de relatórios e acompanhamento de dados logísticos.
\end{itemize}

\vspace{-5pt}


\subsection*{Analista de Dados – Completa Atacadista (Set/2010 – Nov/2013)}
\begin{itemize}
    \item Liderança técnica na migração de processos manuais em Excel para o \textbf{ERP SAP Business One}.
    \item Automação de relatórios com \textbf{SQL e VBA}, reduzindo em até 70\% o tempo de execução de rotinas.
    \item Criação de \textbf{dashboards estratégicos} de vendas e estoque para suporte à tomada de decisão.
    \item Desenvolvimento de integrações de sistemas usando \textbf{Python e Shell Script}, aumentando a confiabilidade dos dados corporativos.
\end{itemize}

\vspace{-5pt}

% --- Formação ---
\section*{Formação Acadêmica}

\vspace{-5pt}


\textbf{MBA em Data Science \& Analytics} — USP/ESALQ \hfill \textbf{(Mai/2025 – Dez/2026)} (em andamento) \\

\vspace{-5pt}


\textbf{Bacharelado em Física} — USP/IFSC \hfill \textbf{(Jan/2017 – Jul/2023)} \\
TCC: Análise de séries temporais financeiras por meio de \textbf{grafos de visibilidade}, com foco em extração de métricas topológicas e modelagem supervisionada.

\vspace{-5pt}


% --- Projetos ---
\section*{Atividades Relevantes}

\subsection*{Análise de Séries Temporais Financeiras via Grafos de Visibilidade e Machine Learning (TCC)}
\textit{USP – Instituto de Física de São Carlos}

\begin{itemize}
    \item Desenvolvimento de um pipeline completo para modelagem de séries temporais financeiras utilizando \textbf{Visibility Graphs}.
    \item Extração automatizada de métricas topológicas (grau, densidade, clustering, conectividade) para composição de vetores de features.
    \item Preparação de dataset tabular e treinamento de modelos supervisionados (\textbf{Random Forest Regressor}), com validação cruzada e análise de importância de variáveis.
    \item Investigação da relação entre propriedades estruturais dos grafos e o comportamento real dos fundos, gerando insights interpretáveis a partir de métricas topológicas.
\end{itemize}

\subsection*{Kenshi Translator Toolkit — Biblioteca Python publicada no PyPI (2025)}
\textit{Projeto independente}

\begin{itemize}
    \item Engenharia reversa completa do formato binário \textbf{.mod} do jogo Kenshi, com desenvolvimento de parser e encoder para leitura, manipulação e reconstrução segura dos arquivos.
    \item Implementação de arquitetura modular com separação entre domínio e infraestrutura, empacotamento e publicação no \textbf{PyPI}.
    \item Biblioteca ativa com usuários externos, distribuída publicamente via \texttt{pip}.
\end{itemize}

\subsection*{GEPAC — Grupo de Estudos em Programação Científica (2019 – 2021)}
\textit{USP – Instituto de Física de São Carlos}

\begin{itemize}
    \item Co-fundador e organizador de grupo técnico voltado à computação científica e programação de alto desempenho.
    \item Elaboração e apresentação de palestras técnicas com material autoral, incluindo:
    \href{https://gepac.github.io/2019-10-09-juliaLang/}{Computação Científica com Julia} e
    \href{https://gepac.github.io/2019-08-21-projeto-membranas/}{Simulação Numérica de Oscilações em Membranas}.
    \item Produção de materiais públicos e projetos educacionais em computação científica, com código e conteúdo publicados em \href{https://gepac.github.io}{gepac.github.io}.
    \item Instrutor de Python científico e contribuidor do \textbf{Grupy-Sanca}.
\end{itemize}

\subsection*{Atuação Técnica Colaborativa em Projetos Acadêmicos (2017 – 2024)}
\textit{USP – IFSC e colaborações informais}

\begin{itemize}
    \item Apoio técnico recorrente a projetos acadêmicos em \textbf{Machine Learning}, \textbf{computação científica} e \textbf{visão computacional}.
    \item Implementação e ajuste de código para TCCs e projetos de pesquisa, incluindo análise de erros, experimentação computacional e validação de resultados.
    \item Colaborações técnicas em contextos de Física Computacional, biofotônica e modelagem em grafos, sem vínculo formal ou autoria científica.
\end{itemize}

% --- Competências ---
\section*{Competências Técnicas}
\begin{itemize}
    \item \textbf{Engenharia de Dados}: pipelines ETL/ELT, integração com sistemas transacionais, versionamento e auditoria de dados.
    \item \textbf{Python \& Performance}: NumPy, SciPy, álgebra linear esparsa, profiling e otimização vetorial.
    \item \textbf{Machine Learning Aplicado}: validação cruzada, métricas, feature engineering, modelagem em grafos.
    \item \textbf{Computação Científica}: simulação numérica, métodos matriciais, análise de desempenho.
    \item \textbf{Ferramentas}: Git, Docker, GitHub Actions, SQL, SAP B1, Power BI.
\end{itemize}

\end{document}
