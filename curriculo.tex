
\documentclass[10pt]{article}

\usepackage{geometry}
\geometry{
 a4paper,
 total={170mm,257mm},
 left=20mm,
 top=20mm,
}
\usepackage[dvipsnames]{xcolor}
\usepackage[portuges]{babel}
\usepackage[utf8]{inputenc}
\usepackage[T1]{fontenc}
\usepackage{hyperref}
\usepackage{fontawesome}

\pagestyle{empty}
\setlength { \parindent }{ 0pt }

\begin{document}

%----------------------------------------------------------------------------------------
%	Nome e Sobrenome, Dados de Contato
%----------------------------------------------------------------------------------------

{\large\bf ALEX SOARES PRESTES \hfill} % Your name at the top
 
\vspace{-5pt}
\rule{\linewidth}{1pt}

{\faEnvelope \ alex{\_}sp\_@hotmail.com \hfill \faHome \ Vl. Falcão - Bauru/SP}

{\faLinkedinSquare \ \href{https://www.linkedin.com/in/alexsprestes/}{alexsprestes} \hfill \faWhatsapp \ (14) 99728-8656 }

{\faGithub \href{https://github.com/AlexPrestes/}{AlexPrestes} \hfill }



%----------------------------------------------------------------------------------------
%	Objetivo: Colocar o cargo que almeja. O objetivo sempre será o nome da vaga que você
%   irá se candidatar.
%----------------------------------------------------------------------------------------
\vspace{10pt}
\centerline{\large\bf Objetivo: Cientista de Dados}

%----------------------------------------------------------------------------------------
%	Resumo Profissional
%----------------------------------------------------------------------------------------
 
\section*{Resumo Profissional}
\vspace{-15pt}
\rule{\linewidth}{1pt}

Profissional com 3 anos de experiência na área de dados como Analista de Dados em empresa do segmento de atacado.
\\

Possuo experiência em interpretar dados, analisar resultados e utilizar técnicas estatísticas.
\\

Vivência em desenvolver e implementar análises de dados, sistemas de coleta de dados e outras estratégias que otimizem a eficiência e a qualidade estatística.
\\

Atualmente busco uma oportunidade como Cientista de Dados.
\\

Durante a minha graduação em Física, tive duas oportunidades de vivência como Cientista de Dados. A primeira foi a participação no grupo GEPAC, que tinha como objetivo apresentar a linguagem Python para cientista, algoritmos, análise e modelagem. A segunda foi em relação o meu TCC foi focado em analisar fundos de investimentos usando diferentes técnicas de análise de dados.
\\

Graduado em Física.
\\

Possuo inglês nível intermediário.


%----------------------------------------------------------------------------------------
%	Formação Acadêmica
%----------------------------------------------------------------------------------------

\section*{Formação Acadêmica}
\vspace{-15pt}
\rule{\linewidth}{1pt}

{\bf Bacharel em Física}, Universidade de São Paulo \\
Instituto de Física de São Carlos \\
Conclusão: Julho/2023 \\
TCC: Análise de fundos de investimento utilizando grafo de visibilidade

%----------------------------------------------------------------------------------------
%	Experiência Profissional
%----------------------------------------------------------------------------------------
 
\section*{Experiência Profissional}
\vspace{-15pt}
\rule{\linewidth}{1pt}

{\bf Analista de Dados} \hfill  Janeiro 2007 - Novembro 2013\\
Completa Atacadista, Bauru, SP

\begin{itemize} \itemsep -2pt % Reduce space between items
    \item SQL e PL/SQL (Criação e Otimização de consultas, desenvolvimento de Stored Procedures);
    \item Excel e VBA (Desenvolvimento de dashboards para diversos departamentos);
    \item Desenvolvimento e manutenção de micro serviços para integração de sistemas (Python, Shell Script e C\#);
    \item Administração de servidores Linux e Windows Server 2008;
\end{itemize}


{\bf Auxiliar administrativo} \hfill  Janeiro 2007 - Setembro 2009\\
Completa Atacadista, Bauru, SP

\begin{itemize} \itemsep -2pt % Reduce space between items
    \item Desenvolvimento de planilhas personalizadas no EXCEL;
    \item Faturamento de pedidos, emissão de nota fiscal e envio para transportadora;
    \item Atendimento ao cliente.
\end{itemize}

%----------------------------------------------------------------------------------------
%	Certificações
%----------------------------------------------------------------------------------------

\section*{Certificações}
\vspace{-15pt}
\rule{\linewidth}{1pt}

\begin{itemize} \itemsep -2pt
    \item \href{https://www.linkedin.com/learning/certificates/f233c38eaee572144d1b8f00a6a1d891dab0712399576d922c470f474a4963a7}{Técnicas Avançadas de Power BI}
    \item \href{https://coursera.org/share/c6d3914afc8a0b3de06c3af0d6ae468c}{Fundamentals of Parallelism on Intel Architecture}
\end{itemize}

%----------------------------------------------------------------------------------------
%	Competências
%----------------------------------------------------------------------------------------

\section*{Competências}
\vspace{-15pt}
\rule{\linewidth}{1pt}

{\bf Linguagens:}
\begin{itemize} \itemsep -2pt
    \item Python: Pytorch, Scikit-Learn, Tensorflow, OpenCV, Pandas.
    \item Power BI: Linguagem M, Linguagem DAX.
    \item Excel: Avançado, Macros e VBA.
    \item SQL, PL/SQL e Stored Procedures.
    \item Outras: C, C++, C\#, Fortran, Julia, Matlab, Shell Script.
\end{itemize}

{\bf Banco de Dados:}
\begin{itemize} \itemsep -2pt
    \item Microsoft SQL Server, MySQL e MongoDB
\end{itemize}

{\bf Ferramentas:}
\begin{itemize} \itemsep -2pt
    \item Docker e Git.
\end{itemize}

{\bf Sistemas Operacionais:}
\begin{itemize} \itemsep -2pt
    \item Windows, Windows Server e Linux.
\end{itemize}

%----------------------------------------------------------------------------------------
%	Idioma
%---------------------------------------------------------------------------------------- 

\section*{Idioma}
\vspace{-15pt}
\rule{\linewidth}{1pt}

{\bf Inglês:}
\begin{itemize} \itemsep -2pt
    \item Leitura: Intermediário
    \item Escrita: Básico
    \item Audição: Intermediário
    \item Fala: Básico
\end{itemize}

%----------------------------------------------------------------------------------------

\end{document}
