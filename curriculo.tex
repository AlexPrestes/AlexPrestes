\documentclass[11pt, a4paper]{article}
\usepackage[utf8]{inputenc}
\usepackage[T1]{fontenc}
\usepackage[default]{opensans}
\usepackage[margin=0.75in]{geometry}
\usepackage{xcolor}
\usepackage{hyperref}
\usepackage{enumitem}
\usepackage{parskip}

% Cores para links
\hypersetup{
    colorlinks=false,
    linkcolor=blue,
    urlcolor=blue,
}

% Configurações de listas
\setlist[itemize]{noitemsep, topsep=0pt, leftmargin=*}
\setlist[enumerate]{noitemsep, topsep=0pt, leftmargin=*}

\begin{document}
% --- Header ---
\begin{center}
    {\Huge \textbf{ALEX SOARES PRESTES}} \\[0.2cm]
    {\large \textbf{Cientista de Dados | Python | Machine Learning | SAP}} \\[0.2cm]
    {\small 
    Bauru, SP | \href{mailto:alex.prestes@outlook.com.br}{alex.prestes@outlook.com.br} | 14 99728-8656 \hfill \\
    \href{https://linkedin.com/in/alexsprestes}{linkedin.com/in/alexsprestes} | \href{https://github.com/AlexPrestes}{github.com/AlexPrestes}
    }
\end{center}

% --- Objetivo ---
\section*{Objetivo}
Atuar como \textbf{Cientista de Dados}, aplicando séries temporais, machine learning e estatística para otimização de processos logísticos e análises financeiras, unindo experiência prática em ERP (SAP) e desenvolvimento de modelos preditivos.

% --- Resumo Profissional ---
\section*{Resumo}
\begin{itemize}
    \item Profissional com trajetória iniciada em \textbf{automação de processos (Excel/VBA)} em 2007 e, desde 2010, experiência em \textbf{análise de dados, ERP e modelagem preditiva}.
    \item Experiência em \textbf{implementação de SAP Business One}, automação de relatórios e criação de dashboards estratégicos.
    \item \textbf{Bacharel em Física (USP)} com sólida formação em estatística, séries temporais e aprendizado de máquina (Random Forest, Redes Neurais, Clustering).
    \item Desenvolvedor ativo: projetos publicados no \textbf{PyPI} e portfólio no GitHub.
    \item MBA em Data Science \& Analytics \emph{(em andamento)} com foco em otimização de modelos preditivos.
\end{itemize}

% --- Experiência Profissional ---
\section*{Experiência Profissional}

\subsection*{Analista de Dados/Gestão de Estoque}
\textbf{Gemmini | Bauru, SP} \hfill Dez/2024–Fev/2025
\begin{itemize}
    \item \textbf{Implementação do SAP}: participei da migração do sistema ERP, focando na integração de dados logísticos.
    \item \textbf{Otimização de estoque}: desenvolvi scripts Python para análise preditiva de demanda de insumos críticos.
\end{itemize}

\subsection*{Analista de Dados}
\textbf{Completa Atacadista | Bauru, SP} \hfill Out/2010–Nov/2013
\begin{itemize}
    \item \textbf{Automação de processos}: macros em VBA e consultas SQL reduziram o tempo de rotinas de 30 para 10 minutos.
    \item \textbf{Implementação do SAP}: migrei 100\% dos dados de Excel para SAP Business One, garantindo integridade na transição.
    \item \textbf{Dashboards estratégicos}: criei relatórios automatizados para análise de estoque e vendas.
\end{itemize}

\subsection*{Auxiliar Técnico em Química}
\textbf{DER – Departamento de Estradas de Rodagem | Bauru, SP} \hfill Set/2009–Set/2010
\begin{itemize}
    \item Atividades de laboratório e controle de qualidade de materiais rodoviários.
    \item Elaboração de relatórios técnicos e análise de resultados experimentais.
    \item Experiência em \textbf{métodos quantitativos e padronização de processos}.
\end{itemize}

\subsection*{Auxiliar Administrativo / Automação de Processos}
\textbf{Completa Atacadista | Bauru, SP} \hfill Jan/2007–Set/2009
\begin{itemize}
    \item Gestão de estoque e pedidos de clientes em Excel.
    \item Reconstrução da base de dados de clientes/produtos usando fórmulas, macros e VBA.
    \item Automação reduziu tempo de processos de 30–40 minutos para 5–10 minutos.
\end{itemize}

% --- Formação Acadêmica ---
\section*{Formação}
\textbf{MBA em Data Science \& Analytics} \hfill Mai/2025–Dez/2026 (previsão)\\
USP/ESALQ \emph{(em andamento)}

\textbf{Bacharelado em Física} \hfill 2017–2023\\
USP/IFSC \\
TCC: \emph{"Classificação de FIIs via grafos de visibilidade e Random Forest"}

% --- Projetos ---
\section*{Projetos Relevantes}

\subsection*{Kenshi Translator Toolkit (2025)}
\begin{itemize}
    \item \textbf{Biblioteca Python} para manipulação de arquivos binários de mods do jogo Kenshi.
    \item Publicada no \textbf{PyPI} com empacotamento profissional e controle de versão semântico.
    \item Implementa \textbf{engenharia reversa} de formato proprietário e serialização binária.
    \item \textbf{Arquitetura em camadas} (domínio, infraestrutura e utilitários).
    \item Tecnologias: Python, manipulação binária (struct), logging, empacotamento (setuptools/uv).
\end{itemize}

\subsection*{TCC: Análise de Fundos Imobiliários (2023–2024)}
\begin{itemize}
    \item Modelagem de séries temporais com grafos de visibilidade e Random Forest (Python, Pandas, Scikit-learn).
    \item Extração e tratamento de dados de informes mensais de FIIs (CVM).
\end{itemize}

\subsection*{GEPAC - Grupo de Estudos (2019–2021)}
\begin{itemize}
    \item Co-fundador de grupo interdisciplinar com foco em programação científica.
    \item Desenvolvimento de algoritmos para simulação de sistemas físicos (Python, NumPy).
\end{itemize}

% --- Competências ---
\section*{Competências Técnicas}
\begin{itemize}
    \item \textbf{Linguagens}: Python (avançado), SQL (avançado), VBA (avançado)
    \item \textbf{Ferramentas}: SAP Business One, Power BI, Excel, Git
    \item \textbf{ML/AI}: Scikit-learn, PyTorch, séries temporais, redes neurais
    \item \textbf{Estatística}: Regressão linear, clustering (K-means, DBSCAN), testes de hipóteses
    \item \textbf{Soft Skills}: Resolução de problemas complexos, comunicação técnica, aprendizado contínuo
\end{itemize}

\end{document}
