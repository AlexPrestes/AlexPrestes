\documentclass[11pt, a4paper]{article}
\usepackage[utf8]{inputenc}
\usepackage[T1]{fontenc}
\usepackage[default]{opensans}
\usepackage[margin=0.75in]{geometry}
\usepackage{xcolor}
\usepackage{hyperref}
\usepackage{enumitem}
\usepackage{parskip}

% Cores para links
\hypersetup{
    colorlinks=false,
    linkcolor=blue,
    urlcolor=blue,
}

% Configurações de listas
\setlist[itemize]{noitemsep, topsep=0pt, leftmargin=*}
\setlist[enumerate]{noitemsep, topsep=0pt, leftmargin=*}

\begin{document}
% --- Header ---
\begin{center}
    {\Huge \textbf{ALEX SOARES PRESTES}} \\[0.2cm]
    {\large \textbf{Cientista de Dados | Python | Machine Learning | SAP}} \\[0.2cm]
    {\small 
   14 99728-8656   -   Bauru, SP \\
    \href{mailto:alex.prestes@outlook.com.br}{alex.prestes@outlook.com.br} \\
    \href{https://linkedin.com/in/alexsprestes}{linkedin.com/in/alexsprestes} \\
    \href{https://github.com/AlexPrestes}{github.com/AlexPrestes}
    }
\end{center}

% --- Objetivo ---
\section*{Objetivo}
Atuar como \textbf{Cientista de Dados}, aplicando expertise em modelagem estatística, análise de séries temporais e machine learning (Random Forest, Redes Neurais) para resolver desafios complexos em finanças ou logística.

% --- Resumo Profissional ---
\section*{Resumo}
\begin{itemize}
    \item Cientista de Dados com \textbf{3+ anos de experiência em análise de dados} e formação interdisciplinar em Física (USP)
    \item Automatizei processos reduzindo tempo em 1/3 com Excel na Completa Atacadista (2009–2013)
    \item Desenvolvi modelo de classificação de FIIs usando grafos de visibilidade e Random Forest (TCC/USP)
    \item Stack técnica: Python (Pandas, Scikit-learn, PyTorch), SQL, SAP Business One, Power BI
    \item MBA em Data Science \emph{(em andamento)} com foco em otimização de modelos preditivos
\end{itemize}

% --- Experiência Profissional ---
\section*{Experiência Profissional}

\subsection*{Analista de Dados}
\textbf{Completa Atacadista | Bauru, SP} \hfill Jan/2009–Nov/2013
\begin{itemize}
    \item \textbf{Automação de processos}: Desenvolvi macros em VBA e consultas SQL que reduziram o tempo de rotinas operacionais de 30 para 10 minutos
    \item \textbf{Implementação do SAP}: Migrei 100\% dos dados de Excel para SAP Business One, garantindo integridade durante a transição
    \item \textbf{Dashboards estratégicos}: Criei relatórios automatizados em Excel para análise de estoque e vendas
\end{itemize}

\subsection*{Analista de Almoxarifado/Dados}
\textbf{Gemmini | Bauru, SP} \hfill Dez/2024–Fev/2025
\begin{itemize}
    \item \textbf{Implementação do SAP}: Participei da migração do sistema ERP, focando na integração de dados logísticos
    \item \textbf{Otimização de estoque}: Desenvolvi scripts Python para análise preditiva de demanda de insumos críticos
\end{itemize}

% --- Formação Acadêmica ---
\section*{Formação}
\textbf{MBA em Data Science \& Analytics} \hfill Mai/2025–Dez/2026 (previsão)\\
USP/ESALQ \emph{(em andamento)}

\textbf{Bacharelado em Física} \hfill 2017–2023\\
USP/IFSC \\
TCC: \emph{"Classificação de FIIS via grafos de visibilidade e Random Forest"}

% --- Projetos ---
\section*{Projetos Relevantes}
\subsection*{TCC: Análise de Fundos Imobiliários (2023)}
\begin{itemize}
    \item Modelagem de séries temporais usando grafos de visibilidade e Random Forest (Python, Pandas, Scikit-learn)
    \item Extração e tratamento de dados de informes mensais de FIIS (CVM)
\end{itemize}

\subsection*{GEPAC - Grupo de Estudos (2019–2021)}
\begin{itemize}
    \item Co-fundador de grupo interdisciplinar com foco em programação científica
    \item Desenvolvimento de algoritmos para simulação de sistemas físicos (Python, NumPy)
\end{itemize}

% --- Competências ---
\section*{Competências Técnicas}
\begin{itemize}
    \item \textbf{Linguagens}: Python (avançado), SQL (avançado), VBA (avançado)
    \item \textbf{Ferramentas}: SAP Business One, Power BI, Excel, Git
    \item \textbf{ML/AI}: Scikit-learn, PyTorch, séries temporais, redes neurais
    \item \textbf{Estatística}: Regressão linear, clustering (K-means, DBSCAN), testes de hipóteses
    \item \textbf{Soft Skills}: Resolução de problemas complexos, comunicação técnica
\end{itemize}

\end{document}
