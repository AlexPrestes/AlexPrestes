\documentclass[11pt, a4paper]{article}
\usepackage[utf8]{inputenc}
\usepackage[T1]{fontenc}
\usepackage[default]{opensans}
\usepackage[margin=0.75in]{geometry}
\usepackage{hyperref}
\usepackage{enumitem}
\usepackage{parskip}

\hypersetup{
    colorlinks=false,
    urlcolor=blue,
}

\setlist[itemize]{noitemsep, topsep=0pt, leftmargin=*}

\begin{document}

% --- Header ---
\begin{center}
    {\Huge \textbf{ALEX SOARES PRESTES}} \\[0.2cm]
    {\large Cientista de Dados | Analista de Dados | Engenheiro de Dados} \\[0.2cm]
    {\small Bauru, SP | \href{mailto:alex.prestes@outlook.com.br}{alex.prestes@outlook.com.br} | (14) 99728-8656} \\
    {\small \href{https://linkedin.com/in/alexsprestes}{linkedin.com/in/alexsprestes} | \href{https://github.com/AlexPrestes}{github.com/AlexPrestes}}
\end{center}

% --- Resumo ---
\section*{Resumo Profissional}
Profissional de dados com experiência em \textbf{estatística aplicada}, \textbf{machine learning supervisionado e não supervisionado}, \textbf{automação de processos}, \textbf{ERP (SAP Business One)} e \textbf{análise de grandes volumes de dados}. 
Formado em \textbf{Física (USP)} e cursando \textbf{MBA em Data Science \& Analytics (USP/ESALQ)}, atuo na construção de soluções que conectam teoria e prática, desde a exploração de dados até a implementação de modelos preditivos e relatórios para tomada de decisão.

% --- Experiência ---
\section*{Experiência Profissional}

\subsection*{Analista de Dados – Gemmini (Dez/2024 – Fev/2025)}
\begin{itemize}
    \item Apoio à transição para o \textbf{SAP Business One}, garantindo consistência na integração de dados do setor logístico.
    \item Desenvolvimento de automações em \textbf{Python e Excel} para relatórios de estoque e indicadores.
    \item Contribuição para a melhoria do acompanhamento de dados operacionais e de estoque.
\end{itemize}

\subsection*{Suporte Técnico N1 – Smart Computadores (2015 – 2016)}
\begin{itemize}
\item Implementação de solução padronizada de \textbf{backup em Linux (Live USB + rsync)}, aumentando a confiabilidade dos dados e eliminando falhas recorrentes.
\item Execução de \textbf{manutenções preventivas}, instalação de softwares e documentação técnica de sistemas corporativos.
\end{itemize}

\subsection*{Analista de Dados – Completa Atacadista (2010 – 2013)}
\begin{itemize}
    \item Liderança na migração de processos manuais em Excel para o \textbf{ERP SAP Business One}.
    \item Automação de relatórios com \textbf{SQL e VBA}, reduzindo em até 70\% o tempo de execução de rotinas.
    \item Criação de \textbf{dashboards estratégicos} de vendas e estoque para suporte à gestão.
    \item Desenvolvimento de integrações de sistemas usando \textbf{Python e Shell Script}, melhorando a confiabilidade dos dados.
\end{itemize}

\subsection*{Auxiliar Administrativo – Completa Atacadista (2007 – 2009)}
\begin{itemize}
    \item Automação de processos em \textbf{Excel/VBA}, reduzindo tempos de execução em até 80\%.
    \item Reconstrução da base de clientes/produtos, melhorando confiabilidade dos dados para operações.
\end{itemize}

% --- Formação ---
\section*{Formação Acadêmica}
\textbf{MBA em Data Science \& Analytics} — USP/ESALQ \hfill (em andamento, conclusão prevista 2026) \\
\textbf{Bacharelado em Física} — USP/IFSC \hfill 2017–2023 \\
TCC: Análise de Fundos Imobiliários utilizando \textbf{grafos de visibilidade e Random Forest} para classificação e métricas aplicáveis a sistemas de recomendação.

% --- Projetos ---
\section*{Projetos Relevantes}
\textbf{Kenshi Translator Toolkit (2025)} — Biblioteca Python publicada no \textbf{PyPI}, com engenharia reversa e serialização binária. \\
\textbf{TCC em Machine Learning (2023–2024)} — Classificação de Fundos Imobiliários com séries temporais, grafos de visibilidade e Random Forest; extração de métricas aplicáveis em classificação e recomendação. \\
\textbf{GEPAC – Grupo de Estudos (2019–2021)} — Co-fundador de grupo interdisciplinar de programação científica; desenvolvimento de algoritmos para simulação e análise de dados em Python (NumPy).

% --- Competências Técnicas ---
\section*{Competências Técnicas}
\begin{itemize}
    \item \textbf{Linguagens}: Python (Pandas, NumPy, Scikit-learn, PyTorch), SQL, VBA
    \item \textbf{Ferramentas}: SAP Business One, Power BI, Excel, Git
    \item \textbf{Machine Learning}: classificação, regressão, clustering, séries temporais, feature engineering, validação de modelos
    \item \textbf{Estatística}: análise exploratória de dados (EDA), testes de hipóteses, regressão linear
    \item \textbf{Dados}: automação de processos, integração de sistemas, qualidade e governança de dados
    \item \textbf{Visualização}: Power BI, matplotlib, relatórios interativos
\end{itemize}

\end{document}