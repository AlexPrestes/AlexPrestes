\documentclass[11pt, a4paper]{article}
\usepackage[utf8]{inputenc}
\usepackage[T1]{fontenc}
\usepackage[default]{opensans}
\usepackage[margin=0.5in]{geometry}
\usepackage{hyperref}
\usepackage{enumitem}
\usepackage{parskip}

\hypersetup{
    colorlinks=true,
    urlcolor=blue,
}

\setlist[itemize]{noitemsep, topsep=0pt, leftmargin=*}

\begin{document}

% --- Header ---
\begin{center}
    {\Huge \textbf{ALEX SOARES PRESTES}} \\[0.2cm]
    {\large Machine Learning Engineer | Scientific Computing | High-Performance Python} \\[0.2cm]
    {\small Bauru, SP | \href{mailto:alex.prestes@outlook.com.br}{alex.prestes@outlook.com.br} | (14) 99728-8656} \\
    {\small \href{https://linkedin.com/in/alexsprestes}{linkedin.com/in/alexsprestes} | \href{https://github.com/AlexPrestes}{github.com/AlexPrestes}}
\end{center}

% --- Resumo ---
\section*{Resumo Profissional}
Engenheiro de Machine Learning com forte atuação em pipelines de dados, otimização de performance em Python e computação científica. Experiência prática em modelagem, automação, integrações com ERP (SAP B1) e construção de fluxos analíticos escaláveis. Formado em Física (USP), com base sólida em álgebra linear, processamento eficiente e engenharia de dados. Atualmente cursando MBA em Data Science \& Analytics (USP/ESALQ), atuo no desenvolvimento de soluções ponta a ponta — da estruturação de datasets à validação, versionamento e documentação de pipelines.


% --- Experiência ---
\section*{Experiência Profissional}

\subsection*{Machine Learning Engineer – Projeto Freelancer (Nov/2025)}
\textit{Remoto, Brasil}
\begin{itemize}
    \item Reestruturei um pipeline completo de link prediction em grafos biológicos, aumentando a performance em \textbf{163x} e a escalabilidade em \textbf{329x}.
    \item Otimizei o cálculo das features estruturais para 34 milhões de pares de nós, executando todo o processamento em apenas \textbf{4,4 segundos} via vetorização e álgebra linear esparsa.
    \item Estruturei todo o fluxo de dados (ingestão → dataset → treino/teste), incluindo um \textbf{K-Fold customizado} para grafos sem data leakage e validação estatística (teste t pareado).
    \item Documentei o pipeline de ponta a ponta, garantindo reprodutibilidade, auditabilidade e facilidade de evolução.
\end{itemize}

\subsection*{Analista de Dados – Gemmini (Dez/2024 – Fev/2025)}
\begin{itemize}
    \item Apoio à transição para o \textbf{SAP Business One}, garantindo consistência na integração de dados do setor logístico.
    \item Desenvolvimento de automações em \textbf{Python e Excel} para relatórios de estoque e indicadores.
    \item Contribuição para a melhoria do acompanhamento de dados operacionais e de estoque.
\end{itemize}

\subsection*{Analista de Dados – Completa Atacadista (2010 – 2013)}
\begin{itemize}
    \item Liderança na migração de processos manuais em Excel para o \textbf{ERP SAP Business One}.
    \item Automação de relatórios com \textbf{SQL e VBA}, reduzindo em até 70\% o tempo de execução de rotinas.
    \item Criação de \textbf{dashboards estratégicos} de vendas e estoque para suporte à gestão.
    \item Desenvolvimento de integrações de sistemas usando \textbf{Python e Shell Script}, melhorando a confiabilidade dos dados.
\end{itemize}

% --- Formação ---
\section*{Formação Acadêmica}

\textbf{MBA em Data Science \& Analytics} — USP/ESALQ \hfill 2024–2026 (em andamento) \\

\textbf{Bacharelado em Física} — USP/IFSC \hfill 2017–2023 \\
TCC: Análise de séries temporais financeiras por meio de \textbf{grafos de visibilidade}, com foco em extração de métricas topológicas e modelagem em \textbf{Random Forest}.

% --- Projetos ---
\section*{Projetos Relevantes}

\subsection*{Análise de Fundos de Investimento via Grafos de Visibilidade e Aprendizado de Máquina (TCC)}
\textit{USP – Instituto de Física de São Carlos}

\begin{itemize}
    \item Desenvolvimento de um pipeline completo para modelagem de séries temporais financeiras utilizando \textbf{Visibility Graphs} para transformar séries em estruturas topológicas analisáveis.
    \item Extração de métricas estruturais de grafos (grau, densidade, clustering, conectividade) para composição do vetor de features.
    \item Construção automática de vetores de features e preparação de dataset tabular para modelagem supervisionada.
    \item Treinamento de modelos (\textbf{Random Forest Regressor}) com validação cruzada, análise de importância de variáveis e otimização de hiperparâmetros.
    \item Análise da relação entre as propriedades estruturais dos grafos e o comportamento real dos fundos, gerando insights interpretáveis a partir de métricas topológicas.
\end{itemize}

\textbf{Tecnologias:} Python, Pandas, NumPy, Scikit-learn, NetworkX, Matplotlib.
\vspace{-5pt}

\subsection*{Kenshi Translator Toolkit — Biblioteca Python publicada no PyPI (2025)}
\textit{Projeto independente}
\begin{itemize}
    \item Engenharia reversa completa do formato binário \textbf{.mod} do jogo Kenshi, criando um parser estruturado para leitura e extração de registros, metadados e diálogos.
    \item Implementação de um \textbf{encoder binário} capaz de reconstruir arquivos válidos, incluindo flags, referências, instâncias e manipulação segura de padrões textuais (WordSwap).
    \item Projeto modular com \textbf{arquitetura limpa} (domínio/infraestrutura), empacotamento e publicação no \textbf{PyPI}.
    \item Biblioteca ativa com usuários externos e distribuição via PyPI (instalação \texttt{pip}).
\end{itemize}

\textbf{Repositório:} \href{https://github.com/AlexPrestes/kenshi-translator-toolkit}{github.com/AlexPrestes/kenshi-translator-toolkit} \\
\textbf{PyPI:} \href{https://pypi.org/project/kenshi-translator-toolkit/}{pypi.org/project/kenshi-translator-toolkit} \\
\textbf{Tecnologias:} Python, struct, dataclasses, parsing binário, arquitetura modular, PyPI packaging.
\vspace{-5pt}
\subsection*{GEPAC – Grupo de Estudos em Programação Científica (2019–2021)}
\textit{USP – Instituto de Física de São Carlos}

\begin{itemize}
    \item Co-fundador do grupo; apresentei projetos técnicos em sala de aula, incluindo:
        \href{https://gepac.github.io/2019-10-09-juliaLang/}{Quem é Julia?} (computação científica de alto desempenho)  
        e  
        \href{https://gepac.github.io/2019-08-21-projeto-membranas/}{Oscilações em Membranas} (simulação numérica da equação de ondas).
    \item Instrutor de Python científico utilizando material do \textbf{Grupy-Sanca},
          do qual também fui \href{https://python-sanca.github.io/curso-python/epilogue.html#contribuidores}{contribuidor}.
    \item Desenvolvimento de materiais públicos e projetos de simulação, publicados em:
          \href{https://gepac.github.io}{gepac.github.io}.
\end{itemize}

\textbf{Tecnologias:} Python, NumPy, SciPy, Matplotlib, Julia, simulação numérica.
\vspace{-5pt}

% --- Competências Técnicas ---
\section*{Competências Técnicas}
\begin{itemize}
    \item \textbf{Engenharia de Dados}: construção e otimização de pipelines ETL/ELT, modelagem dimensional (staging, marts, estrela), integração com sistemas transacionais, versionamento de dados.
    \item \textbf{Python \& Performance}: Pandas, NumPy, SciPy, álgebra linear esparsa, profiling, otimização vetorial, automação de pipelines.
    \item \textbf{SQL}: consultas analíticas, modelagem, criação de pipelines de transformação; experiência com SAP B1 e integrações corporativas.
    \item \textbf{Orquestração e DevOps}: Git, Docker, GitHub Actions; familiaridade com Airflow/dbt (mencionar aqui ajuda sem mentir).
    \item \textbf{Machine Learning e Grafos}: feature engineering, validação cruzada, métricas topológicas, Visibility Graphs.
    \item \textbf{Visualização e BI}: Power BI, matplotlib.
\end{itemize}


\end{document}
